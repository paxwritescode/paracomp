\section{Постановка задачи}
Требуется решить задачу Дирихле для уравнения Пуассона:
\begin{equation}
	\label{eq:dirichlet}
	\frac{\partial^2 u}{\partial x^2} + \frac{\partial^2 u}{\partial y^2} = 
(2 - y^2) cos x
\end{equation}
в области $G$ с границей $\Gamma$, заданной следующим образом:
\begin{equation}
	\label{eq:G}
	\overline{G} = \{(x, y) \: | \: 0 \leqslant x \leqslant \pi, \: 0 \leqslant y \leqslant 1\};
\end{equation}
На границе прямоугольника $G$ известно значение решения (т.е. задано граничное условие первого рода):
\begin{equation}
	\label{eq:boundary}
	\begin{cases}
		u(0, y) = y^2, \: 0 \leqslant y \leqslant 1; \; u(x, 0) = 0, \: 0 \leqslant x \leqslant \pi \\
		
		u(\pi, y) =-y^2, \: 0 \leqslant y \leqslant 1; \; u(x, 1) = cos x, \: 0 \leqslant x \leqslant \pi
	\end{cases}
\end{equation}

При решении задачи требуется использовать метод Зейделя с одномерной столбцовой декомпозицией для параллелизации вычислений. 

Также необходимо провести анализ данной параллельной реализации, выполнив исследования следующих зависимостей:
\begin{enumerate}
	\item зависимость времени выполнения от размера изображения;
	\item зависимость времени выполнения от числа параллельных процессов/потоков;
	\item зависимость ускорения от числа параллельных процессов/потоков;
	\item зависимость эффективности параллелизации от числа параллельных процессов/потоков.
\end{enumerate}
\section{Алгоритм метода и условия его применимости}
\subsection{Метод Зейделя для задачи Дирихле для уравнения Пуассона}
Для численного решения задачи \eqref{eq:dirichlet} используется равномерная разностная сетка с шагами $h_x$ и $h_y$ по координатным направлениям. В узлах сетки $(x_i, y_i)$ вводится сеточная функция 
\[
v_{i, j} \approx u(x_i, y_j), \; i = \overline{1, N_x}, j = \overline{1, N_y},
\]
при этом значения решения в граничных узлах считаются известными.

Аппроксимация оператора Лапласа центральными разностями первого порядка приводят к пятиточечному разностному шаблону:
\begin{equation}
	\label{eq:diff_template}
	\frac{v_{i + 1, j} - 2v_{i, j} + v_{i - 1, j}}{h_x^2} + \frac{v_{i, j + 1} - 2v_{i, j} + v_{i, j - 1}}{h_y^2} = f_{i, j}
\end{equation}
Выразим из \eqref{eq:diff_template} $v_{i, j}$:
\begin{equation}
	\label{eq:v_ij}
	v_{i, j} = \frac{\frac{v_{i + 1, j} + v_{i - 1, j}}{h_x^2} + \frac{v_{i, j - 1} + v_{i, j + 1}}{h_y^2} - f_{i, j}}{\frac{2}{h_x^2} + \frac{2}{h_y^2}}.
\end{equation}
Итерационная формула Зейделя для разностной аппроксимации уравнения Пуассона имеет вид:
\begin{equation}
	\label{eq:Seidel}
		v_{i, j}^{(k + 1)} = \frac{\frac{v_{i + 1, j}^{(k)} + v_{i - 1, j}^{(k + 1)}}{h_x^2} + \frac{v_{i, j - 1}^{(k)} + v_{i, j + 1}^{(k + 1)}}{h_y^2} - f_{i, j}}{\frac{2}{h_x^2} + \frac{2}{h_y^2}},
\end{equation}
где $k$ - номер итерации, а $h_x$ и $h_y$ - шаги сетки по координатным направлениям.

Итерационный процесс продолжается до выполнения условия сходимости:
\[
\max_{i, j} \left| v_{i, j}^{(k + 1)} -v_{i, j}^{(k)} \right| < \varepsilon,
\]
где $\varepsilon$ - заданная точность.
\subsection{Одномерная столбцовая декомпозиция}
На \autoref{fig:columnardecomposition} показана иллюстрация одномерной столбцовой декомпозиции расчётной области, используемая для параллельной реализации метода Зейделя. Исходная область  разбивается на вертикальные подобласти, каждая из которых обрабатывается отдельным параллельным процессом. Заштрихованные области соответствуют фиктивным ячейкам, предназначенным для хранения граничных значений сеточной функции, получаеммых от соседних процессов.

Декомпозиция области позволяет выполнять пересчёт значений сеточной функции во внутренних узлах каждой подобласти параллельно. Для обеспечения корректности вычислений на каждой итерации осуществляется обмен граничными значениями между соседними подобластями с использованием средств MPI. 
\begin{figure}
	\centering
	\includegraphics[width=0.25\linewidth]{fig/columnar_decomposition.png}
	\caption{Одномерная столбцовая декомпозиция}
	\label{fig:columnardecomposition}
\end{figure}

\subsection{Условия применимости метода}
Метод Зейделя применим для решения систем линейных алгебраических уравнений, к которым сводятся путём дискретизации эллиптические дифференциальные уравнения второго порядка. Для задачи Пуассона, аппроксимированной на равномерной сетку, матрица системы обладает диагональным преобладанием и является симметричной и положительно определённой.
\subsection{Технология MPI}
Для параллельной реализации алгоритма в данной работе используется стандарт MPI (Message Passing Interface), предназначенный для программирования распределённых вычислений. В рамках работы MPI используется для распределения подобластей между процессами, обмена граничными значениями сеточной функции и синхронизации итераций метода Зейделя.
\section{Предварительный анализ задачи}
\section{Проверка условий применимости метода}
ОТ РУКИ
\section{Тестовый пример с детальными расчётами}
\includepdf[pages=-]{Test_example_lab2}
\section{Подготовка контрольных тестов для иллюстрации метода}
\section{Модульная структура программы}
\section{Численный анализ решения задачи}
\section{Выводы}
\section{Примечания}



